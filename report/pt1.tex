\section{Overview}
In the table below we list the planned and actual time usage for each of the project parts. We define these project parts as given in the assignment text.

\begin{table}[h!]
	\begin{center}
	    \begin{tabulary}{\textwidth}{|l|l|l|}
	    \hline
	    \textbf{Project part} & \textbf{Estimated time usage} & \textbf{Actual time usage} \\ \hline
	    Pt. 1: Project plan & 12,5 hours & 12 hours  \\ \hline
	    Pt. 2: System test plan & 18 hours & 18 hours \\ \hline
	    Pt. 3: General design & 80 hours & 35 hours \\ \hline
	    Pt. 4: Implementation and testing & 110 hours & 74 hours \\ \hline
	    Pt. 5: Final reporting & 10 hours & 8 hours \\ \hline
	    \end{tabulary}
	\end{center}
	\caption{Time usage}
	\label{table1}
\end{table}


\section{Description}
As we can see from \ref{table1} our estimated time usage for part 1 and 2 are quite accurate compared to actual time usage. This is because our group had already completed most of part 1 and 2 when we made the time usage estimations for the entire project. If we take a look at part 3, we can see that we overestimate the time usage by 45 hours. The reason for this is most likely lack of information about the project, and limited experience with time estimation amongst group members. The implementation and testing phase was closer to our target estimation in percentage. However, we notice a significant deviation, possibly due to effective task delegation between group members.  


