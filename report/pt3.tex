We feel that this project has given us a good insight into how a larger project with several units should be executed. Even though we only had TDT4140 and didnt't do the entire "fellesprosjekt" we feel that the planning and execution of our part of the project was a good representation of how we immagine a proper development process is executed. Although we feel that overall the execution of the project have been very smooth, there are parts which we realize that we should have done differently. This brings us to the next section.

\section{Things to avoid in future projects}
Overall we feel that the project have been a very smooth process for all of us, and we have all worked well together towards our common goals. Therefore there are not a great deal of things we wish to avoid in future projects, and those that we do wish to avoid are almost exclusively due to missunderstandings. \\
For instance one thing we wish to avoid in future projects is missunderstandings concerning meeting times. A typical example is when we make a verbal arangment to start at 10, and some of us are certain that 10 is 10:15 rather than 10 sharp. This is a quite common missunderstanding at NTNU though, and not something that have lead to any problems. \\
Another important thing to avoid in future projects is lack of communication. This to ensure that we do not have problems with parts of the implementation not interfacing with eachother due to differences in design. Again this did not create a real problem that was beyond a quick fix, but none the less an important part to have in mind for future projects. \\
Thirdly we could have made a more clear agreement on the distribution of the workload. This to avoid that a single individual is assigned a seemingly simple task that in reality turns out to be quite complex and time demanding. \\
Lastly we should have planned more ahead in terms of working enviroment. Since we consistently met at the same times we should have ensured that we had a good place to sit and work, rather than resorting to one of the larger computer labs. A proper room to work in would most likely result in a more focused enviroment with less noise which in turn would lead to a more efficient session.

So with these realizations it's time to reflect a bit on what we did well and should carry on with us into the next project we participate in.

\section{Things to bring into future projects}
One of the things we are most content with as a group is the use of a common platform for versioncontrol. We decided to use git via the github servers rather than svn. This choice enabled us to work independently over several different platforms. This was a necessity for us since all of us use rather different operating systems and sdk's. It also ensured that changes were easily merged along the way rather than a manual merge in the end of the project, which often results in alot of work before the project is functional. But versioncontrol in itself is not a magical solution, and as previously stated we did have some problems with compatability even with versioncontrol. \\
Another good practice we used was to divide the project into smaller and more managable parts. This made it fairly easy to assign managable tasks to the members of the group. This breakdown of the project worked very well, even though we had a couple of parts that proved to be more extensive than anticipated. \\
This fragmentation of the project also allowed us to better utilize the different skills within the group. Since we (mostly) had small and managable parts we could assign these in a more specific manner to the members best equip to handle those perticular challanges. This ensured a more efficient way of working and a good utilization of our different skills. \\
Lastly we have, as mentioned, been working on the project at specific set times every week, which have ensured a steady progression of the project. This way we could handle challanges in the project in a timely manner and  ensure that we were meeting our set deadlines.


\section{How to prepare better for future projects}
In this section we will reflect on how we can improve our preparations both in ways which we didn't think of earlier and to fix the problems listed in section 3.1.
First of all we can better our communication. One way to do this is to have a meeting at the start of the project to set some ground rules to avoid misinterpretations. During this meeting a common system for communication can also be agreed upon, instead of ending up using different channels to reach different members of the group outside of the meeting times. \\
Secondly we should have spent a bit more time in the planning phase of the project to create a more detailed analysis of the workload.  This is to prevent occasions when a seemingly small task suddenly reveals itself to be more complex and time demanding than previously anticipated. This would also enable us to better decide which member should be responsible for which task before we start with the implementation. This was done during the project, but we realize now that we should  have spent more time on these preparations to save us time later in the project. \\
Thirdly we could also have spent more time on PU3 and made the design a bit more detailed. This would have saved us some time during the end of the implementation stage, where we on occasion realized that something rather should have been implemented in way X instead of Y. A more detailed design plan would probably prevent a few of those issues from arising. \\
Finally there are the demands to the system. These demands should have been discussed more among the group members. Almost all of the demands have been mentioned at some point to some extension, but there have been cases where we have made assumptions about demands without discussing them formally in the group. The result from this is that we have had occasions where the assumptions differ and two modules that are supposed to interact are designed with different assumptions in mind. This most often result in time being spent to first clarify the demands and then redesign some of the modules. With a more coherent list where the demands are clearly defined as we interpret them this extra work could have been completely avoided. \\
All in all we can conclude that to prepare better for future projects we need to spend more time in the planning stage of the project, thus saving time during the actual implementation. 
